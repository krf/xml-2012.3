\documentclass[a4paper,11pt]{scrartcl}
\usepackage[left=2.5cm, right=3cm]{geometry}              % flexible and complete interface to document dimensions
\usepackage[latin1]{inputenc}                             % font encoding for umlauts
\usepackage[ngerman]{babel}                               % language settings
\usepackage[T1]{fontenc}                                  % T1 font encoding
\usepackage{amsmath}                                      % AMS mathematical facilities
%\usepackage{amssymb}                                     % defines symbol names for math symbols in the fonts MSAM and MSBM
\usepackage[bitstream-charter]{mathdesign}                % mathematical fonts to fit with Bitstream Charter
\usepackage{courier}                                      % replaces the current typewriter font for Adobe Courier
\usepackage{listings}                                     % source code printer
\usepackage{color}                                        % adding colors
\usepackage{fancyhdr}                                     % extensive control of page headers and footers
\usepackage{booktabs}                                     % publication quality tables
\usepackage{xcolor}                                       % driver-independent color extensions   
\usepackage{tikz}                                         % creating graphics programmatically
\usepackage{float}                                        % improved interface for floating objects
\usepackage{subfig}                                       % figures broken into subfigures
\usepackage{multirow}
\usepackage[linesnumbered,ruled,vlined]{algorithm2e}      % floating algorithm environment with algorithmic keywords

\definecolor{lightgray}{rgb}{.95,.95,.95}

\lstset{backgroundcolor=\color{lightgray},
        basicstyle=\ttfamily\fontsize{9pt}{9pt}\selectfont\upshape,
				commentstyle=\rmfamily\slshape,
				keywordstyle=\rmfamily\bfseries\color{black},
				captionpos=b,
				showstringspaces=false,
				breaklines=true,
				frame=lines,
				tabsize=2,
				aboveskip={\baselineskip}
}

\renewcommand{\labelenumi}{(\arabic{enumi})}
\renewcommand{\labelenumii}{(\alph{enumii})}
\renewcommand{\algorithmcfname}{Algorithmus}

\subject{Dokumentation}
\title{XML Technologien Projekt}
\author{David Bialik $\cdot$ Kevin Funk\\ Jan Kostulski $\cdot$ Konrad Reiche $\cdot$ Andr� Zoufahl}
\date{\today}
\publishers{\vskip 2ex -- Gruppe 3 --}

\begin{document}

\pagestyle{fancy}
\maketitle

\section{Einleitung}

Dieses Dokument beschreibt die L�sung des Softwareprojekts f�r den Kurs XML Technologien Sommersemester 2012 bearbeitet von Gruppe 3. Aufgabenstellung war es �ber den Dienst \emph{gpsies.com} Informationen �ber verschiedene Routen in Form von XML Dokumenten zu beschaffen, diese unter Nutzung von semantischer Anreicherung zu erweitern und schlussendlich auf einer selbst entwickelten Weboberfl�che anzubieten.

Im weiteren werden wir auf Implementierungsdetails eingehen, erl�utern welche Probleme aufgetreten sind und deren L�sungen diskutieren. Am Ende gibt es eine Anleitung zur Verwendung der Anwendung. Zun�chst wird die Systemarchitektur unserer L�sung skizziert.

\section{Systemarchitektur}

% TODO: All

\section{Implementierung}

\subsection{Crawler}

Das Script \emph{crawler.py} umfasst folgende Funktionen

\begin{lstlisting}
usage: crawler [-h] [-c] [-e] [-s] [-p]

XML Crawler for gpsies.com

optional arguments:
  -h, --help        show this help message and exit
  -c, --crawl       Crawl from gpsies.com, parse result pages
  -e, --extend      Crawl from gpsies.com, extend non-augmented tracks with
                    track details
  -s, --statistics  Print database statistics
  -p, --prune       Clear database, remove all non-augmented tracks
\end{lstlisting}

Der Crawler implementiert zwei Hauptaktionen (\emph{crawl, extend}), um die Datenbank mit XML-Daten von \emph{gpsies.org} zu f�llen.

\subsubsection{Crawl}
\begin{itemize}
  \item Es werden Resultpages mit jeweils 100 Tracks von \emph{gpsies.org} abgefragt
\footnote{Beispiel-URL: http://www.gpsies.org/api.do?key=API\_KEY\&country\=DE\&limit=100\&resultPage=1}
  \item Davon werden die einzelnen \emph{<track/>}-Elemente geparst
  \item Diese werden einzeln als Dokument in die Datenbank geschrieben
  \item Der Dokumentenname ist hierbei die Track-FileID
\end{itemize}

\subsubsection{Extend (Augment)}
\begin{itemize}
  \item Hier werden sukzessiv nicht-augmentierte Tracks aus der Datenbank angereichert
  \item Pro Track werden die Trackdetails von \emph{gpsies.org} geholt
\footnote{Beispiel-URL: http://www.gpsies.org/api.do?key=API\_KEY\&fileId=cimmxugjixiyzakj\&trackDataLength=250}
  \item Danach wird der Track durch die detaillreichere Version in der Datenbank ersetzt
\end{itemize}

\subsection{SPARQL-Abfrage}

% TODO: Konrad

\subsection{Webserver}

% TODO: Kevin, Andre

\section{Test-Umgebung}

\subsection{Referenzsystem}

Unsere Anwendunge wurde haupts�chlich auf folgender Plattform getestet:

\begin{itemize}
  \item Ubuntu 12.04

  \begin{itemize}
    \item Python 2.7.3
    \item BaseX 7.0.2
  \end{itemize}
\end{itemize}

\subsection{Tests}

% TODO: Kevin

\section{Installationsanleitung}

\subsection{Abh�ngigkeiten}

\begin{itemize}
  \item BaseX (http://basex.org/)
  \item Tornado Web Server (http://www.tornadoweb.org/)
  \item Python LXML (http://lxml.de)
  \item SPARQL Endpoint interface to Python (http://sparql-wrapper.sourceforge.net/)
\end{itemize}

\subsubsection{Quick install (f�r Debian-basierte Systeme)}

Getestet auf Ubuntu 12.04.

\begin{lstlisting}
* apt-get install basex python-tornado python-lxml python-sparqlwrapper
\end{lstlisting}


\subsection{Datenbank aufsetzen}

% TODO: Woher Datenbankdump
Zun�chst muss der Datenbankdump (\emph{default-*.zip}) nach \emph{\$HOME/BaseXData/} kopiert werden um sie sp�ter importieren zu k�nnen.

Starten der BaseX-Datenbank

\begin{lstlisting}
$ basexserver
\end{lstlisting}

Erstmaliges Erstellen der Datenbank

\begin{lstlisting}
$ basexclient

# Notiz: Der Standardlogin ist admin:admnin

$ > create database default
\end{lstlisting}

Importieren unseres Datenbankdumps:
\begin{lstlisting}
$ > restore default
\end{lstlisting}

\subsection{Ausf�hren des Crawlers (optional)}

Zum Ausf�hren des Crawlers muss eine Internetverbindung bestehen und die Verf�gbarkeit von \emph{gpsies.org} gew�hrleistet sein.
Au�erdem muss eine BaseX-Server-Instanz f�r die Datenbankanbindung laufen.

Danach kann der Crawler folgenderma�en gestartet werden:
\footnote{Notiz: Alle ausf�hrbaren Dateien liegen im Projektorder \emph{src/}}

\begin{lstlisting}
# Parse result pages from gpsies.org
$ ./crawler.py -c
\end{lstlisting}

Der Crawler l�uft mit diesem Parameter solange, bis er mindestens $100000$ Tracks in der Datenbank gespeichert hat. Diesen Tracks fehlen dann aber noch Detaillinformationen, wie die Startpunktaddresse, welche aber f�r die Suchmaske auf dem Webformular gebraucht werden.

Um diese Informationen zu bekommen muss danach der Crawler mit folgenden Parameters gestartet werden.

\begin{lstlisting}
# Augment tracks with information from gpsies.org
$ ./crawler.py -e
\end{lstlisting}

Bei diesem Aufruf fragt der Crawler f�r jeden Track in der Datenbank die Detaillinformationen auf \emph{gpsies.org} ab und reichert damit die jeweiligen Tracks an.

\subsection{Anreicherung mit Point-Of-Interests (optional)}

% TODO: Konrad?

\subsection{Webserver starten}

Nachdem die Datenbank bef�llt ist, kann der Webserver gestartet werden.

\begin{lstlisting}
# Start web server
$ ./web.py
\end{lstlisting}

Dies startet einen Tornado-Webserver auf der Adresse \emph{http://localhost:8888} welche mit dem Browser angesteuert werden kann.

\section{Benutzerdokumentation}

\end{document}
